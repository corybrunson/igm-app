\documentclass{article}

\begin{document}

The Consensus Plot compares the extent of consensus among the panelists on a question to the extent of clear opinion on that question. The {\bf consensus} is defined as the difference between the panelists who agree with each other (whether in agreement or disagreement with each statement) and those who disagree, as proportions of the number of pairings among those who were not uncertain. This statistic is adapted from the rank-correlation coefficient \(\tau\); the pairwise agreement and disagreement counts are called the {\bf concordance} and {\bf discordance}.

This time, encode \(i\)'s response to \(j\) as agreement (\(r_{ij}=+1\)), disagreement (\(r_{ij}=-1\)), or uncertainty (\(r_{ij}=0\)). (These calculations ignore strong (dis)agreement.) The confidence weights are calculated as before.

Write \((i,j)\) if panelist \(i\) responded to question \(j\). The {\em uncertainty} of question \(j\) is \(\sum_{(i,j)}{c_{ij}(1-|r_{ij}|)}/\sum_{(i,j)}{c_{ij}}\), the ratio of uncertain responses to all responses. The {\em consensus} of question \(j\) is calculated, analogously to the \(\tau_a\) statistic, as \(x_j=\sum_{(i,j),(i',j)}{c_{ij}r_{ij}c_{i'j}r_{i'j}}/{comb(\sum_{(i,j)}{|c_{ij}r_{ij}|},2)}\), the ratio of the difference between the numbers of agreements and of disagreements (concordance minus discordance) to the number of pairs of clear responses (\(comb(a,b)=a!/(b!(a-b)!)\)). Finally, the minimum value of \(x_j\)---when the opining panelists are evenly split---is \(-1/(n-1)\), where \(n\) is the number of opining panelists. To correct for this, the plot uses the adjusted values \(((n-1)x_j+1)/n\), which range from 0 to 1.

\end{document}