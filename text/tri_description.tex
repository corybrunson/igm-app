\documentclass{article}

\begin{document}

The Agreement Plot arranges the survey questions on the {\bf 2-simplex}, which is the triangle connecting the points \((1,0,0)\), \((0,1,0)\), and \((0,0,1)\) in 3-space and its interior. Their positions are encoded as {\bf homogeneous coordinates}: The numbers of panelists who agree with, disagree with, and are uncertain about each question form a triple \((a,d,u)\), possibly weighted by strength of (dis)agreement or confidence. These coordinates determine a line through the origin (which would be the same line if the coordinates were scaled together). Since the coordinates are all nonnegative, the line passes through the 2-simplex; this intersection locates the question in the plot. (This scheme is adapted from correspondence analysis.)

For each panelist \(i=1,\ldots,N\) and each question \(j=1,\ldots,M\), encode \(i\)'s response to \(j\) as strong agreement (\(r_{ij}=+1+\sigma\)), agreement (\(r_{ij}=+1\)), uncertainty (\(r_{ij}=0\)), disagreement (\(r_{ij}=-1\)), or strong disagreement (\(r_{ij}=-1-\sigma\)), where \(\sigma\in[0,1]\) is a tuning parameter controlled by the user. (\(r_{ij}\) is not defined if \(i\) left no opinion on \(j\).) The panelists also recorded their confidence \(C_{ij}\in[1,10]\) in their answers. We logit-transform these values to
\[C'_{ij}=\log({C_{ij}}/{(11-C_{ij})}),\]
take \(Z\)-scores
\[Z_{ij}=(C'_{ij}-m_r)/s_r\]
specific to each response \(r\) (with \(m_r\) and \(s_r\) the response-specific mean and standard deviation), and calculate {\em confidence weights} \(c_{ij}=1+\gamma Z_{ij}\), where \(\gamma\in[0,1/3]\) is a tuning parameter, also controlled by the user.

The {\em agreement} on question \(j\) is \(a(j)=\sum_{r_{ij}>0}{c_{ij}r_{ij}}\), the sum of the agreeing responses, weighted by both confidence and strength of agreement; the {\em disagreement} \(d(j)\) is defined analogously. The {\em uncertainty} is \(u(j)=\sum_{r_{ij}=0}{c_{ij}}\), the weighted sum of the uncertain responses. The standardized coordinates \((a(j),d(j),u(j))/(d(j)+a(j)+u(j))\) are then projected to two dimensions via linear transformations:
\begin{eqnarray*}
x(j) &= &({\sqrt{3}}/{2})\times u(j) - {1}/{2\sqrt{3}} \\
y(j) &= &({1}/{2})\times a(j) - ({1}/{2})\times d(j)
\end{eqnarray*}

\end{document}